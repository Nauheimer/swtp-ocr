\author{Tadd\"aus Nauheimer}
\chapter{Einf\"uhrung}

\section{Intro}

Willkommen zum Guide der Software/Website "Automatic Exam Correction".
In diesem Dokument wird darauf eingegangen welche M\"oglichkeiten es gibt die Software zu verwenden und wie dies in der Umsetzung aussieht.

\section{Requirements}

Die Anforderungen an das Nutzersystem sind recht niedrig, es wird lediglich ein Browser ben\"otigt.

Gebraucht wird:
\begin{itemize}
	\item Gescannte/Digitale Dokumente als PDF oder Bild (png, jpeg)
		\begin{enumerate}
			\item Musterl\"osung (Ausgef\"uhlt oder Leer)
			\item Ausgef\"ullte Formulare zur Korrektur
		\end{enumerate}
	\item Eine Internetverbindung (bei langsamen Verbindungen lohnt es sich auch Geduld mitzubringen)
\end{itemize}

Sind diese Dinge vorhanden kann es auch direkt losgehen.
%Was wie mit der Software umgegangen wird, 
Wie die Bedienung der Software funktioniert, wird im n\"aechsten Abschnitt (Step-by-Step Guide) erkl\"art. 
