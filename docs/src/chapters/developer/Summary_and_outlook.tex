\author{Bautrelle Fotso}
\graphicspath{ {./src/chapters/developer/media/} }

\newpage
\section{Summary and outlook}
Some results obtained during this project are summarize in the table below.

\begin{itemize} \bfseries

\item \textbf{Models built on ready-to-use datasets} \hfill \break

\end{itemize}
  \begin{tabular} {|p{4cm}||p{3cm}|p{3cm}|p{3cm}|} 
    \hline 
    
    \multicolumn{4}{|c|}{model prediction results} \\
    \hline
    Models      &Accuracy [\%] &Loss [\%] &Used Network\\
    
    \hline
    MNIST              &98.24     &8.84     &DFFNN\\
    
    EMNIST-digits      &98.96     &4.769    &2D-CNN\\
    
    EMNIST-balanced    &83.33     &58.29    &2D-CNN\\
    
    EMNIST-byMerge     &89.16     &32.13    &2D-CNN\\
    
    EMNIST-letters     &89.28     &47.32    &DFFNN\\
    
    \hline
   \end{tabular}

   \noindent
This table shows the different prediction rate achieved by evaluating each model created on test data.
It is observable that, especially the EMNIST-digits model has almost 100\% accuracy and 
the lowest error rate compared to other models. 
The results obtained with the numerical models (MNIST and EMNIST-digits) let conclude that, the two types of network(2D-CNN and DFFNN), 
on which all ready-to-use datasets have been built, are both very performant.
The other models, have a considerable high loss.  
However it is interesting to notice that, each of these models has an accuracy of more than 80\%. 
This let predict that, they can have a high recognition for characters on which they have not trained.

\begin{itemize} \bfseries
    \item \textbf{Model built on own dataset(SWTP-AI)}
\end{itemize}

\noindent
As already mentioned in this work, the tool tesseract-ocr, in which the LSTM network is integrated, was used to deploy and train 
the SWTP-AI dataset instead of building a RNN network. 
The results of the deployment are in the table below: \hfill \break 

\begin{tabular} {|p{3cm}||p{3cm}|p{3cm}|} 
    \hline 
    
    \multicolumn{3}{|c|}{Deployment results on the  SWTP-AI dataset} \\
    \hline
    labelled data     &training data    &Accuracy [\%]\\ 
    \hline
    3837              &3773     	    &99.99\%\\
    
    \hline
   \end{tabular} \hfill \break 

\noindent
This table gives the number of data which resulted after the deployment of SWTP-AI with tesseract-ocr.
The difference between the number of characters labelled for the deployment and the resulting data which
have been used for the training is very low. 
This means that, the SWTP-AI dataset have been good deployed and is ready for use. 
Unfortunately the recognition with tesseract have not been sufficiently performed. But the SWTP-AI dataset can be 
increased for a better recognition.  \hfill \break 

\noindent
The features of the OCR using AI could not be completely performed in this project.
However, the recognition achieved gave satisfactory results for the digits and letters prediction as well 
as the building of a custom dataset.
This could be a good starting point to develop the project further and get better results and more interesting features
for the recognition of oral exams, language translation and better recognition of alphabetical and special characters. 
